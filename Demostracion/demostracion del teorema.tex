\documentclass{article}
\usepackage{amsmath, amsthm, amssymb}
\usepackage{geometry}
\usepackage{fancyhdr}

\pagestyle{fancy}
\fancyhf{}
\fancyhead[L]{}
\fancyfoot[R]{Ailema Matos, Raúl R. Espinosa}
\fancyfoot[C]{\thepage}

\title{Igualdad de las integrales inferior y superior como criterio de integrabilidad de Riemann}
\author{
Ailema Matos C121,
Raúl R. Espinosa C122 
}

\newtheorem{theorem}{Teorema}

\begin{document}
\maketitle

\section*{}

En este documento se demuestra rigurosamente mediante un análisis detallado de las definiciones y propiedades fundamentales de la integración de Riemann 
un teorema que nos permite asegurar que una funcion sea integrable según Riemann.\\

\begin{theorem}[Condición necesaria y suficiente de integrabilidad de Riemann en funciones reales]
Sea \( f: \mathbb{R} \to \mathbb{R} \) entonces se cumple que \[f \in \mathcal{R}[a, b] \Longleftrightarrow \underline{I} = \underline{\int_a^b} f(x) \, dx = \overline{\int_a^b} f(x) \, dx = \overline{I}\]\\
\end{theorem}

\textbf{Propiedades y definiciones:}

\begin{enumerate}
    \item \textbf{def 1:}  \(\overline{I} = inf \{ S(f, P): P \in P[a, b] \} \)  
    \item \textbf{def 2:}  \(\underline{I} = sup\{ s(f, P): P \in P[a, b] \} \)
    \item \textbf{Propiedad 1:} Sean \(P_1, P_2:  P1 \in P[a, b] \wedge P_2 \in P[a,b] \), se cumple que:
    \[
    P_2 \supseteq P_1  \Longleftrightarrow\ S(f, P_2) \leq S(f, P_1)
    \]
    \[
    P_2 \supseteq P_1\Longleftrightarrow\ s(f, P_2) \geq s(f, P_1)
    \]
\end{enumerate} 

\begin{proof}
Para demostrar este teorema, procedemos en dos partes:
\begin{enumerate}
    \item \textbf{Necesidad:} 
Demostremos que  \(f \in \mathcal{R}[a, b] \implies \underline{I} = \overline{I} \)

    \textbf{Afirmacion 1:} \(\overline{I} = s(f, P_{sup}) \implies \forall  P_i: P_i \in P[a,b]\) se tiene que \(P_{sup} \supseteq P_i\).\\
    \textbf{Afirmacion 2:} \(\underline{I} = S(f, P_{inf}) \implies \forall  P_i: P_i \in P[a,b]\) se tiene que \(P_{inf} \supseteq P_i\).
\begin{proof}
Demostremos la afirmacion 1:\\

Sea \(\overline{I} = s(f, P_{sup})\) por la \textbf{def 2} se tiene que \(s(f, P_{sup}) = sup\{ s(f, P): P \in P[a, b] \}\) de donde por definicion de supremo se tiene para todo \(  s(f, P_i): s(f, P_i) \in \{ s(f, P): P \in P[a, b] \}\)  se cumple que \(s(f, P_{sup}) \geq s(f, P_i)\) note que para todo \(P_i \in P[a,b]\) se tiene que \(s(f, P_{sup}) \geq s(f, P_i)\) y usando la \textbf{Propiedad 1} se tiene que para todo \(P_i \in P[a,b]\) se cumple que \( P_{sup} \supseteq P_i\).
\end{proof}

\begin{proof}
Demostremos la afirmacion 2:\\

Sea \(\underline{I} = S(f, P_{inf})\) por la \textbf{def 1} se tiene que \(S(f, P_{inf}) = inf\{ S(f, P): P \in P[a, b] \}\) de donde por definicion de infimo se tiene para todo \(  S(f, P_{inf}): S(f, P_i) \in \{ S(f, P): P \in P[a, b] \}\)  se cumple que \(S(f, P_{inf}) \leq S(f, P_i)\) note que para todo \(P_i \in P[a,b]\)se tiene que \(S(f, P_{inf}) \leq S(f, P_i)\) y usando la \textbf{Propiedad 1} se tiene que para todo \(P_i \in P[a,b]\) se cumple que \( P_{inf} \supseteq P_i\).
\end{proof}

 Supongamos que \( f \) es integrable según Riemann. Entonces, por definición
\[
lim(\sigma(f, P,\{\xi_i\})) = I
\]
existe un único número \( I \) tal que para todo \( \epsilon > 0 \), existe una partición \( P_\epsilon \in P[a, b]\)  donde para todo \( P: P \in P[a, b] \wedge P \supseteq P_\epsilon\) se cumple que:
    \[
    |\sigma(f, P,\{\xi_i\}) - I| < \epsilon
    \]\\
Por la \textbf{Afirmacion 1} se tiene que \(P_{sup} \supseteq P_\epsilon\) de donde 
\[
|\sigma(f, P_{sup},\{\xi_i\}) - I| < \epsilon
\]
tomando \(\{\xi_i\}\) como \(\{m_i\}\) el conjunto de los minimos de cada intervalo de la particion \(P_{sup}\) entonces se cumple que
\[
|\sigma(f, P_{sup},\{m_i\}) - I| < \epsilon
\]
Note que \(\sigma(f, P_{sup},\{m_i\}) = \Sigma m_i\Delta x_i = s(f, P_{sup})\) de donde
\[
|s(f, P_{sup}) - I| < \epsilon
\]
\[
|\overline{I} - I| < \epsilon
\]
por definicion de limite se tiene que \(lim\overline{I} = I\).\\

Por la \textbf{Afirmacion 2} se tiene que \(P_{inf} \supseteq P_\epsilon\) de donde 
\[
|\sigma(f, P_{inf},\{\xi_i\}) - I| < \epsilon
\]
tomando \(\{\xi_i\}\) como \(\{M_i\}\) el conjunto de los maximos de cada intervalo de la particion \(P_{inf}\) entonces se cumple que
\[
|\sigma(f, P_{inf},\{M_i\}) - I| < \epsilon
\]
Note que \(\sigma(f, P_{inf},\{M_i\}) = \Sigma M_i\Delta x_i = S(f, P_{inf})\) de donde
\[
|S(f, P_{inf}) - I| < \epsilon
\]
\[
|\underline{I} - I| < \epsilon
\]
por definicion de limite se tiene que \(lim\underline{I} = I\).\\

Note que de \(lim\overline{I} = I\) y \(lim\underline{I} = I\) se tiene que \(lim(\overline{I} - \underline{I}) = 0\) y como \(\overline{I} - \underline{I}\) es una constante, entonces \(\overline{I} - \underline{I} = 0\).\\

Por tanto, concluimos que \(\overline{I} = \underline{I}\), y queda demostrado que 
\[f \in \mathcal{R}[a, b] \implies \underline{I} = \overline{I} \]

    \item \textbf{Suficiencia:} Demostremos que \(\underline{I} = \overline{I} \implies f \in \mathcal{R}[a, b]\)
\textbf{Afirmacion 3:} \(s(f, P) \leq \sigma(f, P,\{\xi_i\}) \leq S(f, P)\)
\begin{proof}
Demostremos la afirmacion 3:\\

Sean \(\{m_i\}\) el conjunto de los minimos de cada intervalo de \(P\) y \(\{M_i\}\) el conjunto de los maximos de cada intervalo de \(P\).
Note que 
\[
m_i \leq f(\xi_i) \leq M_i
\]
\[
m_i \Delta x_i \leq f(\xi_i)\Delta x_i \leq M_i\Delta x_i
\] 
\[
\Sigma m_i \Delta x_i \leq \Sigma f(\xi_i)\Delta x_i \leq \Sigma M_i\Delta x_i
\]
\[
s(f, P) \leq \sigma(f, P,\{\xi_i\}) \leq S(f, P)
\]
\end{proof}
Supongamos que  \(\underline{I} = \overline{I}\) y sea \(I\): \(I = \underline{I} = \overline{I}\).\\

Por definicion de \(\overline{I}\) (infimo) y de \(\underline{I}\) (supremo) se tiene que 
\[
S(f, P) - \overline{I} >= 0 \wedge lim (S(f, P) - \overline{I}) = 0
\]

\[
\underline{I} - s(f, P) \geq 0 \wedge lim (s(f, P) - \underline{I}) = 0
\]
Por definicion de limite para todo \(x > 0, y > 0\), tomando \(\epsilon = max(x, y)\) se tienen las desigualdades 
\[
0< |S(f, P) - \overline{I} | < x \leq \epsilon
\]
\[
0< |s(f, P) - \underline{I} | < y \leq \epsilon 
\]
que son equivalentes a
\[
0< S(f, P) - \overline{I}  < \epsilon
\]
\[
0<  \underline{I} -  s(f, P)  < \epsilon
\]
Por la \textbf{Afirmacion 3} tenemos que 
\[
s(f, P) \leq \sigma(f, P,\{\xi_i\}) \leq S(f, P)
\]
restando \(I\)
\[
s(f, P) -\underline{I} \leq \sigma(f, P,\{\xi_i\}) - I \leq S(f, P) - \overline{I}
\]
usando las desigualdades anteriores llegamos a 
\[
-\epsilon < s(f, P) -\underline{I} \leq \sigma(f, P,\{\xi_i\}) - I \leq S(f, P) - \overline{I} < \epsilon
\]
que se puede escribir como
\[
0 <|\sigma(f, P,\{\xi_i\}) - I |< \epsilon
\]
Como la desigualdad se cumple para todo \(\epsilon > 0\), por definicion de limite se tiene que 
\[
lim(\sigma(f, P,\{\xi_i\})) = I
\]
donde por definicion de Riemann integrable se tiene que \(f \in \mathcal{R}[a, b]\). Por tanto, queda demostrado que \( \overline{I} = \underline{I} \implies f \in \mathcal{R}[a, b] \)


\end{enumerate}

\end{proof}

\end{document}