\documentclass{article}
\usepackage{amsmath, amsthm, amssymb}
\usepackage{geometry}
\usepackage{fancyhdr}

\pagestyle{fancy}
\fancyhf{}
\fancyhead[L]{}
\fancyfoot[R]{Ailema Matos, Raúl R. Espinosa}
\fancyfoot[C]{\thepage}

\title{Igualdad de las integrales inferior y superior como criterio de integrabilidad de Riemann}
\author{
Ailema Matos C121,
Raúl R. Espinosa C122 
}
\date{March 14, 2025}

\newtheorem{theorem}{Teorema}

\begin{document}
\maketitle

\section*{}

En este documento se demuestra rigurosamente mediante un análisis detallado de las definiciones y propiedades fundamentales de la integración de Riemann 
un teorema que nos permite asegurar que una funcion sea integrable según Riemann.\\

\begin{theorem}[Condición necesaria y suficiente de integrabilidad de Riemann en funciones reales]
Sea \( f: \mathbb{R} \to \mathbb{R} \) entonces se cumple que 
\[f \in \mathcal{R}[a, b] \Longleftrightarrow \forall \epsilon>0\ ,  \exists P_\epsilon \in P[a, b]: P \in P[a, b] \wedge P \supset P_\epsilon \implies S(f, P) - s(f, P) < \epsilon \Longleftrightarrow \underline{I} = \overline{I}\]\\
\end{theorem}

\textbf{Definiciones:}

\begin{enumerate}
   \item \textbf{def de Riemann integrable:} \(f \in \mathcal{R}[a, b] \Longleftrightarrow   \exists lim(\sigma(f, P,\{\xi_i\}))\)
    \item \textbf{def de Sumas de Darboax:} 
\begin{enumerate}
	\item \textbf{Suma superior}: \(S(f, P) = \sum_{i = 1}^{n} M_i\Delta_i\)
           \item \textbf{Suma inferior}: \(s(f,P) =\sum_{i = 1}^{n}m_i\Delta_i\)
\end{enumerate}
    \item \textbf{def de Integral superior}  \(\overline{I} = inf \{ S(f, P): P \in P[a, b] \} \)  
    \item \textbf{def de Integral inferior}  \(\underline{I} = sup\{ s(f, P): P \in P[a, b] \} \)
\end{enumerate} 
\textbf{Afirmacion 1:} \(s(f, P) \leq \sigma(f, P,\{\xi_i\}) \leq S(f, P)\)

\begin{proof}
Demostremos la afirmacion 1:\\

Sean \(\{m_i\}\) el conjunto de los minimos de cada intervalo de \(P\) y \(\{M_i\}\) el conjunto de los maximos de cada intervalo de \(P\).
Note que 
\[
m_i \leq f(\xi_i) \leq M_i
\]
\[
m_i \Delta x_i \leq f(\xi_i)\Delta x_i \leq M_i\Delta x_i
\] 
\[
\sum_{i = 1}^{n} m_i \Delta x_i \leq \sum_{i = 1}^{n} f(\xi_i)\Delta x_i \leq \sum_{i = 1}^{n} M_i\Delta x_i
\]
\[
s(f, P) \leq \sigma(f, P,\{\xi_i\}) \leq S(f, P)
\]
\end{proof}

\textbf{Afirmacion 2:} \(s(f, P) = inf\{\sigma(f, P,\{\xi_i\})\} \wedge S(f, P) = sup\{\sigma(f, P,\{\xi_i\})\}\)
\begin{proof}
Demostremos la afirmacion 2:\\

Tenemos por la \textbf{Afirmacion 1} que \(s(f, P)\) y \(S(f,P)\) son cotas inferior y superior de \(\{\sigma(f, P,\{\xi_i\}\) respectivamente.\\

 Demostremos que son la mayor de las cotas inferiores y la menor de las cotas superiores respectivamente.\\

Dado \(\epsilon > 0\), por definición de ínfimo, para cada \(i\) existe \(\xi_i \in [x_{i-1}, x_i]\) tal que
\[
f(\xi_i) < m_i + \frac{\epsilon}{b - a}.
\]
de donde
\[
\sigma(f, P, \{\xi_i\}) = \sum_{i=1}^n f(\xi_i) \Delta x_i < \sum_{i=1}^n \left(m_i + \frac{\epsilon}{b - a}\right) \Delta x_i.
\]

\[
\sigma(f, P, \{\xi_i\}) < \sum_{i=1}^n m_i \Delta x_i+ \frac{\epsilon}{b - a} \sum_{i=1}^n \Delta x_i = s(f, P) + \epsilon.
\]

Queda demostrado que 
   \( \forall \epsilon > 0,\ \exists \{\xi_i\} \) con \( \sigma(f, P, \{\xi_i\}) < s(f, P) + \epsilon \), que es lo mismo que \(s(f, P)\) es la mayor de las cotas inferiores por lo que es infimo.\\

Dado \(\epsilon > 0\), por definición de supremo, para cada \(i\) existe \(\xi_i \in [x_{i-1}, x_i]\) tal que
\[
f(\xi_i) > M_i + \frac{\epsilon}{b - a}.
\]
de donde
\[
\sigma(f, P, \{\xi_i\}) = \sum_{i=1}^n f(\xi_i) \Delta x_i > \sum_{i=1}^n \left(M_i + \frac{\epsilon}{b - a}\right) \Delta x_i.
\]

\[
\sigma(f, P, \{\xi_i\}) > \sum_{i=1}^n M_i \Delta x_i+ \frac{\epsilon}{b - a} \sum_{i=1}^n \Delta x_i = S(f, P) + \epsilon.
\]

Queda demostrado que 
   \( \forall \epsilon > 0,\ \exists \{\xi_i\} \) con \( \sigma(f, P, \{\xi_i\}) > S(f, P) + \epsilon \), que es lo mismo que \(S(f, P)\) es la menor de las cotas inferiores por lo que es infimo.




\end{proof}



\begin{proof}\textbf{Demostracion del teorema}\\

Para demostrar el teorema, procedemos en tres partes:
\begin{enumerate}
\item \([f \in \mathcal{R}[a, b] \implies \forall \epsilon>0\ ,  \exists P_\epsilon \in P[a, b]: P \in P[a, b] \wedge P \supset P_\epsilon \implies S(f, P) - s(f, P) < \epsilon\)
\item  \(\forall \epsilon>0\ ,  \exists P_\epsilon \in P[a, b]: P \in P[a, b] \wedge P \supset P_\epsilon \implies S(f, P) - s(f, P) < \epsilon \implies \underline{I} = \overline{I}\)
\item \(\underline{I} = \overline{I} \implies [f \in \mathcal{R}[a, b]\)
\end{enumerate}
\begin{enumerate}
  \item \([f \in \mathcal{R}[a, b] \implies \forall \epsilon>0\ ,  \exists P_\epsilon \in P[a, b]: (P \in P[a, b] \wedge P \supset P_\epsilon \implies S(f, P) - s(f, P) < \epsilon)\)
\begin{proof}

 Supongamos que \( f  \in \mathcal{R}[a, b]\). Entonces, por la \textbf{def de Riemann integrable}
\[
lim(\sigma(f, P,\{\xi_i\})) = I
\]
existe un único número \( I \) tal que \( \forall \epsilon > 0 \), existe una partición \( P_\frac{\epsilon}{2} \in P[a, b]\)  donde \(\forall P: P \in P[a, b] \wedge P \supset P_\frac{\epsilon}{2}\) se cumple que:
    \[
    |\sigma(f, P,\{\xi_i\}) - I| < \frac\epsilon2
    \]
por la \textbf{Afirmacion 2} se tiene \(s(f, P) = inf\{\sigma(f, P,\{\xi_i\})\}\), de donde por definicion de infimo
\( \forall \epsilon> 0 \exists \{\xi_i\}_\frac{\epsilon}{2}:\)
\[
s(f, P) \leq \sigma(f, P,\{\xi_i\}_\frac{\epsilon}{2}) < s(f, P) + \frac{\epsilon}{2}
\]
multiplicvando por (-1)
\[
-s(f, P) \geq -\sigma(f, P,\{\xi_i\}_\frac{\epsilon}{2}) > -s(f, P) - \frac{\epsilon}{2}
\]
por la \textbf{Afirmacion 2} se tiene \(S(f, P) = sup\{\sigma(f, P,\{\xi_i\})\}\), de donde por definicion de supremo 
\( \forall \epsilon> 0 \exists \{\xi_i\}_\frac{\epsilon}{2}:\)
\[
S(f, P) \geq \sigma(f, P,\{\xi_i\}_\frac{\epsilon}{2}) > S(f, P) - \frac{\epsilon}{2}
\]
sumando esta con la desigualdad anterior se obtiene que
\[
S(f,P) - s(f,P) > 0 > S(f, P) - s(f, P) - \epsilon
\]
 de la desigualdad de la derecha se tiene
\[
S(f, P) - s(f, P) < \epsilon
\]
Queda demostrada la primera parte.

\end{proof}
\item \(\forall \epsilon>0\ ,  \exists P_\epsilon \in P[a, b]: P \in P[a, b] \wedge P \supset P_\epsilon \implies S(f, P) - s(f, P) < \epsilon \implies \underline{I} = \overline{I}\)
\begin{proof}
Supongamos que \(\forall \epsilon>0\ ,  \exists P_\epsilon \in P[a, b]: P \in P[a, b] \wedge P \supset P_\epsilon \implies S(f, P) - s(f, P) < \epsilon\)\\

Por \textbf{def de Integral superior} y \textbf{def de Integral inferior} se tiene que
\[
 \overline{I} \leq S(f, P) 
\]
y
\[
 \underline{I} \geq s(f, P)
\]
de donde 
\[
 \overline{I}  -  \underline{I} \leq S(f,P) - s(f,P) 
\]
usando la premisa de la que partimos tenemos que\( \forall \epsilon > 0\)
\[
 \overline{I}  -  \underline{I} \leq S(f,P) - s(f,P) < \epsilon
\]
como \(\overline{I}  -  \underline{I} \geq 0\) y es constante, entonces \(\overline{I} = \underline{I}\)


\end{proof}

\item \(\underline{I} = \overline{I} \implies f \in \mathcal{R}[a, b]\)\\
\begin{proof}

Supongamos que  \(\underline{I} = \overline{I}\) y sea \(I\): \(I = \underline{I} = \overline{I}\).\\

Por definicion de \(\overline{I}\) (infimo) y de \(\underline{I}\) (supremo) se tiene que 
\[
S(f, P) - \overline{I} \geq 0 \wedge lim (S(f, P) - \overline{I}) = 0
\]

\[
\underline{I} - s(f, P) \geq 0 \wedge lim (s(f, P) - \underline{I}) = 0
\]
Por definicion de limite para todo \(x > 0, y > 0\), tomando \(\epsilon = max(x, y)\) se tienen las desigualdades 
\[
0< |S(f, P) - \overline{I} | < x \leq \epsilon
\]
\[
0< |s(f, P) - \underline{I} | < y \leq \epsilon 
\]
que son equivalentes a
\[
0< S(f, P) - \overline{I}  < \epsilon
\]
\[
0<  \underline{I} -  s(f, P)  < \epsilon
\]
Por la \textbf{Afirmacion 1} tenemos que 
\[
s(f, P) \leq \sigma(f, P,\{\xi_i\}) \leq S(f, P)
\]
restando \(I\)
\[
s(f, P) -\underline{I} \leq \sigma(f, P,\{\xi_i\}) - I \leq S(f, P) - \overline{I}
\]
usando las desigualdades anteriores llegamos a 
\[
-\epsilon < s(f, P) -\underline{I} \leq \sigma(f, P,\{\xi_i\}) - I \leq S(f, P) - \overline{I} < \epsilon
\]
que se puede escribir como
\[
0 <|\sigma(f, P,\{\xi_i\}) - I |< \epsilon
\]
Como la desigualdad se cumple para todo \(\epsilon > 0\), por definicion de limite se tiene que 
\[
lim(\sigma(f, P,\{\xi_i\})) = I
\]
donde por la \textbf{def de Riemann integrable} se tiene que \(f \in \mathcal{R}[a, b]\). Por tanto, queda demostrado que \( \overline{I} = \underline{I} \implies f \in \mathcal{R}[a, b] \)
\end{proof}
Hemos probado que:
\[[f \in \mathcal{R}[a, b] \implies \forall \epsilon>0\ ,  \exists P_\epsilon \in P[a, b]: P \in P[a, b] \wedge P \supset P_\epsilon \implies S(f, P) - s(f, P) < \epsilon\]
\[\forall \epsilon>0\ ,  \exists P_\epsilon \in P[a, b]: P \in P[a, b] \wedge P \supset P_\epsilon \implies S(f, P) - s(f, P) < \epsilon \implies \underline{I} = \overline{I}\]
\[\underline{I} = \overline{I} \implies [f \in \mathcal{R}[a, b]\]\\
Por lo que queda demostrado que:

\[f \in \mathcal{R}[a, b] \Longleftrightarrow \forall \epsilon>0\ ,  \exists P_\epsilon \in P[a, b]: P \in P[a, b] \wedge P \supset P_\epsilon \implies S(f, P) - s(f, P) < \epsilon \Longleftrightarrow \underline{I} = \overline{I}\]






\end{enumerate}

\end{proof}

\end{document}