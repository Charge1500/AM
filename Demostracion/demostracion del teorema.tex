\documentclass{article}
\usepackage{amsmath, amsthm, amssymb}
\usepackage{geometry}
\usepackage{fancyhdr}

\pagestyle{fancy}
\fancyhf{}
\fancyhead[L]{}
\fancyfoot[R]{Ailema Matos, Raúl R. Espinosa}
\fancyfoot[C]{\thepage}

\title{Igualdad de las integrales inferior y superior como criterio de integrabilidad de Riemann}
\author{
Ailema Matos C121,
Raúl R. Espinosa C122 
}
\date{March 14, 2025}

\newtheorem{theorem}{Teorema}

\begin{document}
\maketitle

\section*{}

En este documento se demuestra rigurosamente mediante un análisis detallado de las definiciones y propiedades fundamentales de la integración de Riemann 
un teorema que nos permite asegurar que una funcion sea integrable según Riemann.\\

\begin{theorem}[Condición necesaria y suficiente de integrabilidad de Riemann en funciones reales]
Sea \( f: \mathbb{R} \to \mathbb{R} \) entonces se cumple que \[f \in \mathcal{R}[a, b] \Longleftrightarrow \forall \epsilon>0  \exists P_\epsilon \in P[a, b]: P \in P[a, b] \wedge P \supset P_\epsilon \implies S(f, P) - s(f, P) < \epsilon \Longleftrightarrow \underline{I} = \overline{I}\]\\
\end{theorem}

\textbf{Definiciones:}

\begin{enumerate}
   \item \textbf{def de Riemann integrable:} \(f \in \mathcal{R}[a, b] \Longleftrightarrow   \exists lim(\sigma(f, P,\{\xi_i\}))\)
    \item \textbf{def de Sumas de Darboax:} 
\begin{enumerate}
	\item jfnd
\end{enumerate}
    \item \textbf{def 1:}  \(\overline{I} = inf \{ S(f, P): P \in P[a, b] \} \)  
    \item \textbf{def 2:}  \(\underline{I} = sup\{ s(f, P): P \in P[a, b] \} \)
\end{enumerate} 

\begin{proof}
Para demostrar este teorema, procedemos en dos partes:
\begin{enumerate}
    \item \textbf{Necesidad:} 
Demostremos que  \(f \in \mathcal{R}[a, b] \implies \underline{I} = \overline{I} \)

\textbf{Afirmacion 1:} \(s(f, P) \leq \sigma(f, P,\{\xi_i\}) \leq S(f, P)\)

\begin{proof}
Demostremos la afirmacion 1:\\

Sean \(\{m_i\}\) el conjunto de los minimos de cada intervalo de \(P\) y \(\{M_i\}\) el conjunto de los maximos de cada intervalo de \(P\).
Note que 
\[
m_i \leq f(\xi_i) \leq M_i
\]
\[
m_i \Delta x_i \leq f(\xi_i)\Delta x_i \leq M_i\Delta x_i
\] 
\[
\sum_{i = 1}^{n} m_i \Delta x_i \leq \sum_{i = 1}^{n} f(\xi_i)\Delta x_i \leq \sum_{i = 1}^{n} M_i\Delta x_i
\]
\[
s(f, P) \leq \sigma(f, P,\{\xi_i\}) \leq S(f, P)
\]
\end{proof}

 Supongamos que \( f \) es integrable según Riemann. Entonces, por la \textbf{def 0}
\[
lim(\sigma(f, P,\{\xi_i\})) = I
\]
existe un único número \( I \) tal que \( \forall \epsilon > 0 \), existe una partición \( P_\epsilon \in P[a, b]\)  donde \(\forall P: P \in P[a, b] \wedge P \supset P_\epsilon\) se cumple que:
    \[
    |\sigma(f, P,\{\xi_i\}) - I| < \epsilon
    \]\\
Por la \textbf{Afirmacion 1} se tiene que \(s(f, P) \leq \sigma(f, P,\{\xi_i\}) \leq S(f, P)\) usando la desigualdad de la derecha 
\[
\sigma(f, P,\{\xi_i\}) \leq S(f, P)
\]
restando \(\overline{I}\) en ambos miembros
\[
\sigma(f, P,\{\xi_i\}) - \overline{I} \leq S(f, P) - \overline{I}
\]
por la \textbf{def 1} tenemos que \(lim (S(f, P)) = \overline{I}\) donde por definicion\\
\(\forall \epsilon > 0 \exists P_\epsilon: \forall P: P \in P[a, b] \wedge P \supset P_\epsilon\) se cumple que
\[
 |S(f, P) - \overline{I}| < \epsilon
\]
usando ademas la desigualdad anterior llegamos a que
\[
|\sigma(f, P,\{\xi_i\}) - \overline{I}| \leq |S(f, P) - \overline{I}| < \epsilon
\]
donde tenemos que  \(\forall \epsilon > 0 \exists P_\epsilon: \forall P: P \in P[a, b] \wedge P \supset P_\epsilon\)
\[
|\sigma(f, P,\{\xi_i\}) - \overline{I}| < \epsilon 
\]
entonces por definicion de limite \(lim( \sigma(f, P,\{\xi_i\})) = \overline{I}\) y como el limite es unico, llegamos a que \(\overline{I} = I\).\\

Por la \textbf{Afirmacion 1} se tiene que \(s(f, P) \leq \sigma(f, P,\{\xi_i\}) \leq S(f, P)\) usando la desigualdad de la izquierda
\[
s(f, P) \leq \sigma(f, P,\{\xi_i\})
\]
restando \(I\) en ambos miembros
\[
s(f, P) - I \leq \sigma(f, P,\{\xi_i\}) - I
\]
usando que \( \forall \epsilon > 0 \), existe una partición \( P_\epsilon \in P[a, b]\): \(\forall P: P \in P[a, b] \wedge P \supset P_\epsilon\) se cumple
    \[
    |\sigma(f, P,\{\xi_i\}) - I| < \epsilon
    \]
tenemos que
    \[
|s(f, P) - I| \leq |\sigma(f, P,\{\xi_i\}) - I| < \epsilon
    \]
de donde se obtiene que  \(\forall \epsilon > 0 \exists P_\epsilon: \forall P: P \in P[a, b] \wedge P \supset P_\epsilon\)
    \[
|s(f, P) - I| < \epsilon
    \]
entonces por definicion de limite \(lim(s(f, P)) = I\) pero por la \textbf{def 2} tenemos que \(lim(s(f, P)) = \underline{I}\) y como el limite es unico, llegamos a que \(I = \underline{I}\).\\

Usando los resultados \(\overline{I} = I\) y \(I = \underline{I}\) tenemos que \(\overline{I} = I = \underline{I}\).\\

Por tanto, concluimos que \(\overline{I} = \underline{I}\), y queda demostrado que 
\[f \in \mathcal{R}[a, b] \implies \underline{I} = \overline{I} \] \\

\item \textbf{Suficiencia:} Demostremos que \(\underline{I} = \overline{I} \implies f \in \mathcal{R}[a, b]\)\\

Supongamos que  \(\underline{I} = \overline{I}\) y sea \(I\): \(I = \underline{I} = \overline{I}\).\\

Por definicion de \(\overline{I}\) (infimo) y de \(\underline{I}\) (supremo) se tiene que 
\[
S(f, P) - \overline{I} \geq 0 \wedge lim (S(f, P) - \overline{I}) = 0
\]

\[
\underline{I} - s(f, P) \geq 0 \wedge lim (s(f, P) - \underline{I}) = 0
\]
Por definicion de limite para todo \(x > 0, y > 0\), tomando \(\epsilon = max(x, y)\) se tienen las desigualdades 
\[
0< |S(f, P) - \overline{I} | < x \leq \epsilon
\]
\[
0< |s(f, P) - \underline{I} | < y \leq \epsilon 
\]
que son equivalentes a
\[
0< S(f, P) - \overline{I}  < \epsilon
\]
\[
0<  \underline{I} -  s(f, P)  < \epsilon
\]
Por la \textbf{Afirmacion 1} tenemos que 
\[
s(f, P) \leq \sigma(f, P,\{\xi_i\}) \leq S(f, P)
\]
restando \(I\)
\[
s(f, P) -\underline{I} \leq \sigma(f, P,\{\xi_i\}) - I \leq S(f, P) - \overline{I}
\]
usando las desigualdades anteriores llegamos a 
\[
-\epsilon < s(f, P) -\underline{I} \leq \sigma(f, P,\{\xi_i\}) - I \leq S(f, P) - \overline{I} < \epsilon
\]
que se puede escribir como
\[
0 <|\sigma(f, P,\{\xi_i\}) - I |< \epsilon
\]
Como la desigualdad se cumple para todo \(\epsilon > 0\), por definicion de limite se tiene que 
\[
lim(\sigma(f, P,\{\xi_i\})) = I
\]
donde por la \textbf{def 0} se tiene que \(f \in \mathcal{R}[a, b]\). Por tanto, queda demostrado que \( \overline{I} = \underline{I} \implies f \in \mathcal{R}[a, b] \)


\end{enumerate}

\end{proof}

\end{document}